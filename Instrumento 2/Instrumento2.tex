\documentclass[a4paper, addpoints, 12pt, answers]{exam}

\usepackage{amsfonts}
\usepackage{amsmath}
\usepackage{amssymb}
\usepackage[brazil]{babel}
\usepackage{fontspec}
\usepackage[margin=3cm,headheight=1.5cm]{geometry}
\usepackage{graphicx}
\usepackage{hyperref}

\setmainfont{Times New Roman}

\pagestyle{headandfoot}

\chead{\includegraphics[height=1.5cm]{Unipampa}}
\cfoot{\thepage}

\setlength{\parindent}{0cm}

\begin{document}

\begin{center}
  \textbf{{\Large Questionário de Perfil do Participante \\ Discussão 2}}
  \par\vspace{20pt}
  \makebox[\linewidth]{Nome:\enspace\hrulefill}
  \makebox[\linewidth]{Local:\enspace\hrulefill \enspace Data:\enspace\hrulefill}
\end{center}

\par\vspace{6pt}

\begin{questions}

\question Considerando que um modelo conceitual de banco de dados deve definir ao menos as entidades de domínio, seus atributos e o número de ocorrências (cardinalidade) possíveis de associações (relacionamento), como você definiria uma gramática básica (DSL) para a sua representação?


\makeemptybox{10cm}

\end{questions}

\end{document}