\documentclass[a4paper, addpoints, 12pt, answers]{exam}

\usepackage{amsfonts}
\usepackage{amsmath}
\usepackage{amssymb}
\usepackage[brazil]{babel}
%\usepackage{fontspec}
\usepackage[margin=3cm,headheight=1.5cm]{geometry}
\usepackage{graphicx}
\usepackage{hyperref}

%\setmainfont{Times New Roman}

\pagestyle{headandfoot}

\chead{\includegraphics[height=1.5cm]{Unipampa}}
%\cfoot{\thepage}

\setlength{\parindent}{0cm}

\begin{document}

\begin{center}
  \textbf{{\Large Discussão 1}}
  \par\vspace{20pt}
  \makebox[\linewidth]{Nome:\enspace\hrulefill}
  \makebox[\linewidth]{Local:\enspace\hrulefill \enspace Data:\enspace\hrulefill}
\end{center}

\par\vspace{6pt}

\begin{questions}

\question Linguagens específicas de domínio com abordagem textual podem ser aplicadas na modelagem conceitual posto que conseguem descrever de forma rápida e concisa determinadas propriedades.
Sendo assim, essas soluções podem ser utilizadas ou mesmo adaptadas de uma forma efetiva no que diz respeito a representação do domínio que modelam.

\begin{checkboxes}
  \choice 1 - Discordo totalmente.
  \choice 2 - Discordo parcialmente.
  \choice 3 - Indiferente.
  \choice 4 - Concordo parcialmente.
  \choice 5 - Concordo totalmente.
\end{checkboxes}

Justifique sua resposta.
\makeemptybox{8cm}

\end{questions}

\end{document}