\documentclass[12pt,a4paper]{article}

\usepackage{amsfonts}
\usepackage{amsmath}
\usepackage{amssymb}
\usepackage[brazil]{babel}
\usepackage{fancyhdr}
\usepackage{fontspec}
\usepackage[margin=3cm,headheight=1.5cm]{geometry}
\usepackage{graphicx}
\usepackage{hyperref}

\setmainfont{Times New Roman}

\pagestyle{fancy}

\fancyhf{}
\chead{\includegraphics[height=1.5cm]{Unipampa}}
%\cfoot{\thepage}

\renewcommand{\headrulewidth}{0pt}

\setlength{\parindent}{0cm}

\begin{document}

\begin{center}
  \textbf{{\Large {Glossário de Termos}}}
  \par\vspace{12pt}
\end{center}

\begin{description}
  \item [Linguagem de Domínio Específico:] É uma linguagem de programação ou linguagem de especificação executável que oferece, por meio de notações e abstrações apropriadas, poder expressivo focado e restrito a um domínio de problema específico \cite{vanDeursen:2000,Fowler:2010}.
  \item [Modelo Conceitual:] É a descrição de um banco de dados de forma independente da implementação usada em sistemas de gerenciamento de banco de dados (SGBDs) \cite{heuser:2009}.
  \item [Abordagem ER:] A abordagem entidade-relacionamento é a técnica \cite{Chen:1976} mais difundida para criação de modelos conceituais de bancos de dados. Esta abordagem foi tão bem aceita que passou a ser considerada uma referência definitiva para a modelagem de banco de dados relacionais. É composta basicamente por um método de diagramação e de conceitos que devem ser respeitados \cite{Cougo:2013}.
  \item [Entidade:] É uma representação de um conjunto de objetos do domínio modelado \cite{heuser:2009}.
  \item [Atributo:] São as características próprias das entidades \cite{heuser:2009}.
  \item [Relacionamento:] É a representação da associação entre os objetos modelados \cite{heuser:2009}.
  \item [Cardinalidade:] Registra o número de ocorrências, possuindo um número mínimo e máximo, com que as entidades podem se associar \cite{heuser:2009}.
  \item [Generalização/Associação:] Associado a este conceito está a ideia de herança de propriedades, ou seja, a atribuição de propriedades particulares a um subconjunto de ocorrências de uma entidade genérica \cite{heuser:2009}. 
\end{description}

{\footnotesize \bibliography{Bibliografia}}
\bibliographystyle{abbrv}

\end{document}