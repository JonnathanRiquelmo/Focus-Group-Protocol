\documentclass[xetex,12pt]{beamer}

\usepackage[brazil]{babel}
\usepackage{xltxtra}
\usepackage{tikz}
\usepackage{../tikz-uml}

\newcommand{\hla}[1]{\textcolor{unipampa-red}{#1}}
\newcommand{\hlb}[1]{\textcolor{unipampa-blue}{#1}}
\newcommand{\hlc}[1]{\textcolor{unipampa-green}{#1}}
\newcommand{\hld}[1]{\textcolor{unipampa-gray}{#1}}

\usetheme{unipampa}

\usetikzlibrary{chains,matrix,positioning,decorations.pathreplacing,calc}
\tikzumlset{font=\sffamily\tiny}

\title[Uma DSL para a Modelagem Conceitual de BDs Relacionais]{Uma Linguagem Específica de Domínio para a Representação de Modelos Conceituais de Bancos de Dados Relacionais}
%\subtitle{Grupo Focal}
\author[]{Jonnathan Riquelmo\\Prof. Dr. Maicon Bernardino\\Prof. Dr. Fábio Basso}
\institute[Unipampa / LESSE]{Universidade Federal do Pampa\\Laboratory of Empirical Studies in Software Engineering (LESSE)}
\date{2019/2}

\begin{document}

\begin{frame}[plain]{}
  \titlepage
\end{frame}

\begin{frame}[t]{Roteiro}
  \tableofcontents
\end{frame}

\AtBeginSection[] {
  \begin{frame}[t]{Roteiro}
    \tableofcontents[currentsection]
  \end{frame}
}

% ################################## %
\section{Abertura}
% ################################## %

\begin{frame}[t]{Preliminares}
  \begin{center}
    \hlb{\textbf{Obrigado por Participar!}}
  \end{center}
  \hlc{O que é um grupo focal?}
  \begin{center}
    Um grupo focal é um método de pesquisa que reúne pessoas em uma sala para prover \textit{feedback} sobre algo.
  \end{center}
  \hlc{Como funciona um grupo focal?}
  \begin{center}
    \includegraphics[scale=0.4]{./Figuras/FocusGroup}
  \end{center} 
\end{frame}

\begin{frame}{Objetivo Principal}
  \begin{center}
    Discutir sobre a gramática de uma linguagem de domínio específico (Domain Specific Language - DSL) para modelagem conceitual de bancos de dados relacionais.
  \end{center}
%  \hlc{O que significa adequabilidade?}
%  \begin{itemize}
%    \item Correto ou apropriado para uma pessoa, propósito ou situação em particular.
%    \item Correto ou aceitável para alguém ou algo.
%  \end{itemize}
%  \hlc{Questão de Investigação?}
%  \begin{center}
%    \hlb{A abordagem de modelagem proposta}\\\hla{é adequada para}\\\hlb{modelagem conceitual de sistemas autoadaptativos?}
%  \end{center}
\end{frame}

\begin{frame}[t]{Primeira Tarefa}
  \begin{enumerate}
    \item Ler o \hlb{Termo de Consentimento Livre e Esclarecido}.
    \begin{itemize}
      \item Se concordar, por favor, complete-o com o nome e assine o documento.
    \end{itemize}   
    \item Responder o \hlb{Questionário de Perfil do Participante}. 
  \end{enumerate}  
\end{frame}

% ################################## %
\section{Fundamentação}
% ################################## %

\begin{frame}[t]{Conceitos Principais}
  \begin{center}
    \begin{tikzpicture}font=\sffamily\footnotesize
%      \draw[dotted] (0,0) grid (12,7);
  
      \node[draw, fill=green!25, minimum height=1cm, text width=3cm, align=center, anchor=north west] (N1) at (0,7) {\textbf{Bancos de Dados Relacionais}};
      \node[align=left, anchor=north west] (N11) at ($(N1.south west)+(0.25,-0.5)$) {coleção de dados com\\relacionamentos predefinidos;};
      \node[align=left, anchor=north west] (N12) at ($(N11.south west)+(0,-0.25)$) {possui estrutura orientada a tabelas.};
      \draw[thick, ->] (N1.south) |- ($(N1.south west)+(0,-0.25)$) |- (N11.west);
      \draw[thick, ->] (N1.south) |- ($(N1.south west)+(0,-0.25)$) |- (N12.west);

      \node[draw, fill=blue!25, minimum height=1cm, text width=3cm, align=center, anchor=north] (N2) at ($(N1.south)+(0,-3)$) {\textbf{Modelagem Conceitual}};
      \node[align=left, anchor=north west] (N21) at ($(N2.south west)+(0.25,-0.5)$) {descreve conceitos e relações;};
      \node[align=left, anchor=north west] (N22) at ($(N21.south west)+(0,-0.25)$) {reflete aspectos do mundo real.};
      \draw[thick, ->] (N2.south) |- ($(N2.south west)+(0,-0.25)$) |- (N21.west);
      \draw[thick, ->] (N2.south) |- ($(N2.south west)+(0,-0.25)$) |- (N22.west);

      \node[draw, fill=red!25, minimum height=1cm, text width=3cm, align=center, anchor=north west] (N3) at ($(N1.south east)+(3.75,-1)$) {\textbf{Modelagem Conceitual de Bancos de Dados Relacionais}};
      \node[align=left, anchor=north west] (N31) at ($(N3.south west)+(0.25,-0.5)$) {abordagem ER;};
      \node[align=left, anchor=north west] (N32) at ($(N31.south west)+(0,-0.25)$) {independente de implementação.};
      \draw[thick, ->] (N3.south) |- ($(N3.south west)+(0,-0.25)$) |- (N31.west);
      \draw[thick, ->] (N3.south) |- ($(N3.south west)+(0,-0.25)$) |- (N32.west);
      \draw[very thick, ->] (N1.east) -- (N3.west);
      \draw[very thick, ->] (N2.east) -- (N3.west);  
    \end{tikzpicture}
  \end{center}
\end{frame}

\begin{frame}[t]{Conceitos Relacionados}
  \begin{description}[Conceitos]
    \item [Entidade:] é o conjunto de objetos da realidade modelada.
    \begin{itemize}
      \item \hlb{Pessoa}, \hlb{Departamento}, \hlb{Produto}
    \end{itemize}
    \\~\\
    \item [Relacionamento:] é o conjunto de associações entre entidades.
    \begin{itemize}
      \item \hlb{Nome} da Pessoa
      \item \hlb{Preço} do Produto
    \end{itemize}
    \\~\\
    \item [Atributo:] é um dado associado a cada ocorrência de uma entidade ou de um relacionamento.
    \begin{itemize}
      \item Muitas Pessoas estão \hlb{lotadas} em um departamento
      \item Um ou muitos Produtos \hlb{pertencem} a uma Marca
    \end{itemize}
  \end{description}
\end{frame}

\begin{frame}[t]{Conceitos Relacionados}
  \begin{description}[Conceitos]
    \item [Cardinalidade] são as ocorrências de uma entidade que podem estar associadas através de um relacionamento.
    \begin{itemize}
      \item Muitas \hlb{(1,n)} Pessoas estão lotadas em um \hlb{(1,1)} departamento
      \item Um ou muitos \hlb{(1,n)} Produtos pertencem a uma \hlb{(1,1)} Marca
      \item Uma Pessoa \hlb{(1,1)} possui nenhum ou muitos \hlb{(0,n)} Dependentes
    \end{itemize} 
    \\~\\
    \item [Generalização/Especialização] é a identificação de subconjuntos de entidades que compartilham características em comum.
    \begin{itemize}
      \item Elemento de caracterização semântica
      \item Conceito de herança
    \end{itemize} 
    
  \end{description}
\end{frame}

\begin{frame}[t]{Diagrama Entidade-Relacionamento}

\begin{figure}
    \centering
    \includegraphics[width=\textwidth]{PDF/Exemplo.jpg}
\end{figure}
  
\end{frame}

\begin{frame}[t]{Requisitos para Modelagem}
  \hlc{A modelagem conceitual de um BD relacional deve:}
  \\~\\
  \begin{description}[]
    \item [Req-1:] identificar as entidades relevantes para o domínio.
    \\~\\
    \item [Req-2:] identificar os atributos que definem as características das entidades ou relacionamentos.
    \\~\\
    \item [Req-3:] identificar as associações entre as entidades.
    \\~\\
    \item [Req-4:] identificar o número de ocorrências possíveis entre as entidades.
  \end{description}
\end{frame}

% ################################## %
\section{Discussões}
% ################################## %

\begin{frame}[t]{Discussão 1}
  \hlc{Considerando que:}
  \begin{itemize}
    \item modelos conceituais de BDs mapeiam conceitos e relações de um domínio;
    \item uma DSL com abordagem textual apresenta um conjunto de sentenças bem definidas por uma sintaxe e semântica própria.
  \end{itemize}
  \vspace{12pt}
  \begin{block}{}
    \begin{center}
      É plausível aplicar uma DSL na modelagem conceitual de BDs relacionais?
    \end{center} 
  \end{block}
\end{frame}

\begin{frame}[t]{Discussão 2}
  \hlc{Considerando que:}
  \begin{itemize}
    \item sim, é plausível aplicar uma DSL na modelagem conceitual de BDs relacionais;
    \item modelos conceituais de BDs relacionais devem especificar:
    \begin{itemize}
          \item as entidades do domínio;
          \item os atributos das entidades;
          \item os relacionamentos entre as entidades;
          \item as cardinalidades dos relacionamentos. 
    \end{itemize}
  \end{itemize}
  \vspace{12pt}
  \begin{block}{}
    \begin{center}
      Como deve ser a gramática aplicada para a\\modelagem conceitual de BDs relacionais?
    \end{center}
  \end{block}
\end{frame}

\begin{frame}[t]{Discussão 2}
\hlc{Exemplo de Diagrama Entidade-Relacionamento}
  \begin{figure}
    \centering
    \includegraphics[width=\textwidth]{PDF/ModeloER.jpg}
\end{figure}
  
\end{frame}

\begin{frame}[t]{Discussão 3}
  \hlc{Considerando:}
  \begin{itemize}
    \item as versões da gramática de modelagem proposta (ver impresso).
  \end{itemize}
  \vspace{12pt}
  \begin{block}{}
    \begin{center}
      Qual gramática para a DSL de modelagem proposta é mais\\adequada para modelagem conceitual de BDs relacionais?
    \end{center}
  \end{block}
\end{frame}

% ################################## %
\section{Fechamento}
% ################################## %

\begin{frame}[t]{Contribuições Finais}
  \begin{center}
    \hlb{Fiquem à vontade para contribuir.}
  \end{center}
  \vspace{46pt}
  \Huge{\centerline{Obrigado por Participar!}}
\end{frame}

\begin{frame}[plain]{}
  \titlepage
\end{frame}

\end{document}